% NOTE: This table uses \multirow; include in your preamble:
% \usepackage{multirow}
% Also requires: \usepackage{booktabs} and \usepackage{float} (for [H]).

\begin{table}[H]
\centering
\small
\caption{Proposal vs. alternative design choices (nowcasting). Lower RMSE is better. ``Best'' denotes the best single run; ``Mean$\pm$Std'' reports performance across multiple seeds/runs.}
\label{tab:proposal-vs-alternatives-nowcast}
\begin{tabular}{llcc}
\toprule
Group & Method & Micro RMSE & Macro RMSE \\
\midrule
\multirow{2}{*}{Proposed} 
& Best single run & 214.263 & 113.391 \\
& Mean$\pm$Std (3 seeds) & 225.075 $\pm$ 8.846 & 120.220 $\pm$ 6.416 \\
\midrule
\multirow{3}{*}{Temporal ablations}
& No lags & 238.946 $\pm$ 20.871 & 128.476 $\pm$ 19.024 \\
& Lag1 only & 228.038 $\pm$ 4.447 & 123.279 $\pm$ 5.908 \\
& Lag1+Lag2 & 241.846 $\pm$ 2.489 & 129.735 $\pm$ 3.635 \\
\midrule
\multirow{2}{*}{Sensor baseline}
& Random sensors (k=10) & 235.761 $\pm$ 7.278 & 125.136 $\pm$ 6.786 \\
\midrule
\multirow{1}{*}{Encoder variant}
& RouteNet-lite (same setup) & 348.617 $\pm$ 28.211 & 198.618 $\pm$ 14.287 \\
\bottomrule
\end{tabular}
\end{table}
